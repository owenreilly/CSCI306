\documentclass[10pt]{article}
\usepackage[margin=.5in]{geometry}
\usepackage{amsmath,amssymb,amsthm} % American Mathematical Society packages for typesetting math.
\usepackage{enumerate} %Allows me to control numbering format for ordered lists.  The enumitem package has fancier options, but you can't use both.
\usepackage{tikz} % We'll use this later in the semester for drawing diagrams with LaTeX code.

% Environment definitions
\newtheorem{theorem}{Theorem}
\newtheorem{claim}{Claim}
\theoremstyle{definition}
\newtheorem{problem}{Problem}

\newcommand{\ZZ}{\mathbb{Z}} % Integers.
\newcommand{\NN}{\mathbb{N}} % Natural Numbers.  NB: There are different definitions of natural numbers based on whether 0 is included.  For this class, 0 is a natural number.
\newcommand{\QQ}{\mathbb{Q}} % Rational numbers.

% Defines a simple command that will label your solution and enclose it in a box.  This will run from page to page automatically.  
\usepackage{tcolorbox}
\tcbuselibrary{breakable}
\newcommand{\solution}[1]{
  \begin{tcolorbox}[breakable,title=Solution \arabic{problem}:,title after break={Solution \arabic{problem}, continued},colframe=black,colback=white,colbacktitle=white,coltitle=black,fonttitle=\bfseries,titlerule=0mm,arc=0mm,lefttitle=-1pt]
    #1
  \end{tcolorbox}
}

% Replace these with your own information.  Remember to use \today.
\title{CSCI 341 - Fall 2024:\\Homework 1}
\author{Owen Reilly}
\date{Due: {\bf On gradescope} Wednesday, Sep. 11, 2024\\Based on problems in Sipser's {\em Introduction to the Theory of Computation}}

\begin{document}
\maketitle


\hrule
\subsection*{Problems}
%%%%%%%%%%%%%%%%%%%%%%%%%%%%%%%%%%%%%%%%%%%

\begin{problem}[Sets]\hfill
\begin{enumerate}[(a)]
\item What is $\mathcal{P}(\{\emptyset,\{\emptyset\}\})$?
\begin{solution}
    $\mathcal{P}(\{\emptyset, \{\emptyset\}\}) = \{\emptyset, \{\emptyset\}, \{\{\emptyset\}\}, \{\emptyset, \{\emptyset\}\}\}$
\end{solution}

\item What is $\mathcal{P}(\emptyset)$?
\item What is $\mathcal{P}\left(\{1,2\}\times\{1,2\}\right)$?
\item What is $\{x \mid x = 2k, k\in \ZZ\}\cap\{x \mid x\text{ is prime}\}$?
\end{enumerate}
\end{problem}

%\solution{
% Uncomment the lines before and after this one and replace this line with your solution (on as many lines as needed).
%}

%%%%%%%%%%%%%%%
\begin{problem}[Set Operators]
If $A = \{1,2,3,4\}, B = \{3,5,7,9\}$, and $C = \{2,3,5,7\}$, what is the following set?
$$D = \left(\{x | x\in A \text{ OR } (x\in B \text{ AND } x\in C)\} \times (C\cap A)\right) - (A\times (B\cup C))$$
\end{problem}

%\solution{
% Uncomment the lines before and after this one and replace this line with your solution (on as many lines as needed).
%}


%%%%%%%%%%%%%%%
\begin{problem}[Relations]
Let $f: A\rightarrow B$ be a function.  Define a relation on $A$ where $x \in A$ is related to $y \in A$ if $f(x) = f(y)$.  Show that this is an equivalence relation.  

Recall that an {\em equivalence class} is a set of elements which are related by an equivalence relation.  Equivalence relations partition their domains into a collection of equivalence classes.  Write down the equivalence classes on $\NN$ generated by the above relation with the function $f:\NN \rightarrow \{0,1,2\}$ given by $f(n) = n \pmod 3$.
\end{problem}

%\solution{
% Uncomment the lines before and after this one and replace this line with your solution (on as many lines as needed).
%}


%%%%%%%%%%%%%%%
\begin{problem}[Proofs]
Find the error in the following proof that $1=2$.  

Consider the equation $a=b$.  Multiply both sides by $a$ to obtain $a^2 = ab$.  Subtract $b^2$ from both sides to get $a^2-b^2 = ab-b^2$.  Factor each side: $(a+b)(a-b) = b(a-b)$, and divide each side by $(a-b)$, to get $a+b = b$.  Finally, let $a$ and $b$ equal $1$, which shows that $2 = 1$.
\end{problem}

%\solution{
% Uncomment the lines before and after this one and replace this line with your solution (on as many lines as needed).
%}


%%%%%%%%%%%%%%%
\begin{problem}[Proofs-Induction]
Find the error in the following proof that all horses are the same color. {\em (Hint: Consider the concrete steps hidden in induction.)}

\begin{claim} In any set of $h$ horses, all horses are the same color.\end{claim}
\begin{proof}
By induction on $h$.
\begin{itemize}
\item Base Case: Let $h = 1$. In any set containing just one horse, all horses are clearly the same color.
\item Inductive Hypothesis: For $k\geq 1$ assume that the claim is true for $h = k$.
\item Inductive Step: Prove that the claim is true for $h = k+1$.  Take any set $H$ of $k+1$ horses.  We show that all the horses in this set are the same color. Remove one horse from this set to obtain the set $H_1$ with just $k$ horses.  By the inductive hypotheses, all the horses in $H_1$ are the same color.  Now replace the removed horse and remove a different one to obtain the set $H_2$.  By the same argument, all the horses in $H_2$ are the same color.  Therefore all the horses in $H$ must be the same color, and the proof is complete.
\end{itemize}
\end{proof}
\end{problem}

%\solution{
% Uncomment the lines before and after this one and replace this line with your solution (on as many lines as needed).
%}

%%%%%%%%%%%%%%%
\begin{problem}[FSAs]
  Draw Finite State Automata for the following languages over the alphabet $\Sigma = \{a,b,c\}$:
  \begin{enumerate}[(a)]
  \item $L = \{ w \mid w[1] = w[-2] \}$ (python-style indexing)
  \item $L = \{ w \mid w \text{ contains at least two $a$s and at least one $b$} \}$
  \end{enumerate}
\end{problem}

%\solution{
% Uncomment the lines before and after this one and replace this line with your solution (on as many lines as needed).
%}

\end{document}
